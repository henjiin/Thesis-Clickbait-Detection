\chapter{Verwandte Arbeiten} % (fold)
\label{cha:verwandte_arbeit}
Soft News und Clickbait in News.
In \enquote{Doing Well and Doing Good: How Soft News and Critical Journalism Are Shrinking
the News Audience and Weakening Democracy–
And What News Outlets Can Do About It} zeigen die Autoren schon 2000 eine zunehmende Abwendung der News Publisher von Hard News in Richtung soft news.
Da hier der Begriff Clickbait noch nicht klar definiert wurde, soll hier die Definition aus \cite{blom2015click} gegeben werden, welche auf dieser Veröffentlichung aufbaut.
\begin{quote}
[Soft news is ] ‘‘typically more personality-centered, less time-bound, more practical, and
more incident-based than other news’’ and associated with ‘‘sensationalism’’, ‘‘human-interest’’ or ‘‘news you can use.’’
\end{quote}
Dabei lässt sich Soft News als Obergruppe von Clickbait verstehen, da es die Merkmale Sensationsmacherei, Menschliches Interesse und News die man gebrauchen kann teilt, jedoch spezifischer ist in der Hinsicht das Clickbait sich primär durch den Überschriftenstil auszeichnet.
Soft News zeigt sich in zu zunehmender Sensationsmacherei, Aufmerksamkeit für  \enquote{Human Interest}, Kriminalität und Katastrophen widerspiegelt. Auch zeigen sie einen Anstieg von Selbstreferenz in News, ein für Clickbait typisches Mittel. 
Linguistische Forschung über Clickbait.
"Click bait: Forward-reference as lure in online news
headlines" \cite{blom2015click} beschäftigt sich detailiertmit der "clickbaitisierung" dänischer News. 
Bei der durchgeführten Studie wurden 100.000 Überschriften 10 verchiedener dänischer News Websiten auf untersucht, untersucht wurde hierbei insbesondere der vorwärts referenzierende Teil von Clickbait.
Es konnten in dem Datensatz 17,2\% der Nachrichten als Clickbait identifiziert werden, wobei der größte Anteil von Clickbait den  sogenannten "soft content" Kategorien Sport, Wetter, \enquote{Lifestyle} und \enquote{Gadget} gefunden wurde. Es ließ sich zudem ein Trend zum Boulevardzeitungsstil und Kommerzialisierung feststellen.
Die Autoren verwendeten zur erkennung von Clickbait hauptsächlich die dänischen Wörter für \enquote{here}, \enquote{this} und \enquote{we} \footnote{her, sådan, derfor, så and dette, denne, dette}.
Damit haben die Autoren die Grundlage für eine automatische Erkennung von Clickbait gebildet, sie erkennen jedoch auch das ihre Ergebnisse nicht völlig  umfassend sind, sondern eher als Teil einer Tendenzbewertung verstanden werten sollten.
Wir stimmen mit der Bewertung, das vorwärtsreferenzierende Überschriften vorrangig das Ziel haben, Leser zum clicken des Artikels zu ködern um Werbeeinnahmen zu erzielen überein.

Psychologische Funktionsweise Clickbaits. 
\cite{berger2012makes} untersuchte Faktoren die zur viralität von Online Content beitragen und kam zum Schluss das Besorgnis- und Wut erregende Geschichten häufiger in Mailinglisten geteilt wurden.  
Das emotionale Geschichten und Überschriften mehr aufmerktsamkeit Erfahren und dementsprechen höhere viralität und einen höheren köderfaktor bei clickbait haben konnte auch in \cite{guerini2015deep} gezeigt werden.
\todo{Paper lesen!}
In journalistischen Artikeln \cite{site:psycholicalreasonsclickbait}) wird häufiger die \enquote{Curiosity Gap} genannt. Dieser Ausdruck bezieht sich auf die Arbeit Loewensteins, welcher in \cite{loewenstein1994psychology} geprägt wurde. Demnach empfinden Menschen oftmals das verlangen eine entstehende Wissenslücke zu schliessen falls sie auf eine solche Treffen, im falle Clickbaits auf das Thema nicht vollständig erklärende, vorwärts referenzierende Überschriften.
Das öffentliche Angelegenheiten bei Lesern deutlich unbeliebter sind als Kriminalität und enternainmeint konnte in \cite{hensinger2013modelling} dargestellt werden.
\todo{MORE}
Listicles, also Artikel die als Listen aufgebaut sind, sind eine Ausprägung von Clickbait. In dem New Yorker Artikel \enquote{A List of Reasons Why Our Brains Love Lists
} argumentiert Konnikova mithilfe von \cite{messner2011unconscious} das sich diese Form von Wissensaufbereitung besonders eignet, um Wissen zu strukturieren und damit zugänglich zu machen. \toto{nochmal}









% chapter verwandte_arbeit (end)
%
%headlines
%journalism
%spam detection
%advertisment
%clickbait
%click fraud 
%text categorization
%history of clickbait
%relevance theory
%listicle
%text sentiment