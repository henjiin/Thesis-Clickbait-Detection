\chapter{Verwandte Arbeiten} % (fold)
\label{cha:verwandte_arbeit}

\section{Linguistische Forschung über Clickbait} % (fold)
\label{sub:section_name}
% section section_name (end)
"Click bait: Forward-reference as lure in online news
headlines" \cite{blom2015click} beschäftigt sich detailiertmit der "clickbaitisierung" dänischer News. 
Bei der durchgeführten Studie wurden 100.000 Überschriften 10 verchiedener dänischer News Websiten auf untersucht, untersucht wurde hierbei insbesondere der vorwärts referenzierende Teil von Clickbait.
Es konnten in dem Datensatz 17,2\% der Nachrichten als Clickbait identifiziert werden, wobei der größte Anteil von Clickbait den  sogenannten "soft content" Kategorien Sport, Wetter, \enquote{Lifestyle} und \enquote{Gadget} gefunden wurde. Es ließ sich zudem ein Trend zum Boulevardzeitungsstil und Kommerzialisierung feststellen.
Die Autoren verwendeten zur erkennung von Clickbait hauptsächlich die dänischen Wörter für \enquote{here}, \enquote{this} und \enquote{we} \footnote{her, sådan, derfor, så and dette, denne, dette}.
Damit haben die Autoren die Grundlage für eine automatische Erkennung von Clickbait gebildet, sie erkennen jedoch auch das ihre Ergebnisse nicht völlig  umfassend sind, sondern eher als Teil einer Tendenzbewertung verstanden werten sollten.
Wir stimmen mit der Bewertung, das vorwärtsreferenzierende Überschriften vorrangig das Ziel haben, Leser zum clicken des Artikels zu ködern um Werbeeinnahmen zu erzielen überein.

\section{Psychologische Funktionsweise Clickbaits} % (fold)
\label{sub:psycho}
\cite{berger2012makes} untersuchte Faktoren die zur Viralität von Online Content beitragen und kam zum Schluss das Besorgnis- und Wut erregende Geschichten häufiger in Mailinglisten geteilt wurden.  
Das emotionale Geschichten und Überschriften mehr Aufmerktsamkeit Erfahren und dementsprechen höhere viralität und einen höheren köderfaktor bei clickbait haben konnte auch in \cite{guerini2015deep} gezeigt werden.
\todo{Paper lesen!}
In journalistischen Artikeln \cite{site:psycholicalreasonsclickbait}) wird häufiger die \enquote{Curiosity Gap} genannt. Dieser Ausdruck bezieht sich auf die Arbeit Loewensteins, welcher in \cite{loewenstein1994psychology} geprägt wurde. Demnach empfinden Menschen oftmals das verlangen eine entstehende Wissenslücke zu schliessen falls sie auf eine solche Treffen, im falle Clickbaits auf das Thema nicht vollständig erklärende, vorwärts referenzierende Überschriften.
Das öffentliche Angelegenheiten bei Lesern deutlich unbeliebter sind als Kriminalität und enternainmeint konnte in \cite{hensinger2013modelling} dargestellt werden.
\todo{MORE}
Listicles, also Artikel die als Listen aufgebaut sind, sind eine Ausprägung von Clickbait. In dem New Yorker Artikel \enquote{A List of Reasons Why Our Brains Love Lists
} argumentiert Konnikova mithilfe von \cite{messner2011unconscious} das sich diese Form von Wissensaufbereitung besonders eignet, um Wissen zu strukturieren und damit zugänglich zu machen. \todo{nochmal}
\cite{ifantidou2009newspaper}

\section{Text Kategorisierung} % (fold)
\label{sub:text_kategorisierung}

\section{Overfitting} % (fold)
\label{sub:overfitting}

\cite{aphinyanaphongs2014comprehensive}
\section{Spamerkennung} % (fold)
\label{sub:spamerkennung}

\section{Twitter} % (fold)
\label{sub:twitter}

\cite{mccord2011spam}
\cite{benevenuto2010detecting}
\cite{kouloumpis2011twitter}
\section{Virale Inhalte} % (fold)
\label{sub:virale_inhalte}
\cite{berger2012makes}

%headlines
%journalism
%spam detection
%advertisment
%clickbait
%text categorization
%history of clickbait
%relevance theory
%listicle
%text sentiment