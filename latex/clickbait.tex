\chapter{Clickbait} 
\label{cha:clickbait}

\section{Soft- und Hard News} % (fold)
\label{sec:soft_und_hard_news}
Soft News und Clickbait in News.
In \enquote{Doing Well and Doing Good: How Soft News and Critical Journalism Are Shrinking
the News Audience and Weakening Democracy–
And What News Outlets Can Do About It} zeigen die Autoren schon 2000 eine zunehmende Abwendung der News Publisher von Hard News in Richtung soft news.
Da hier der Begriff Clickbait noch nicht klar definiert wurde, soll hier die Definition aus \cite{blom2015click} gegeben werden, welche auf dieser Veröffentlichung aufbaut.
\begin{quote}
[Soft news is ] ‘‘typically more personality-centered, less time-bound, more practical, and
more incident-based than other news’’ and associated with ‘‘sensationalism’’, ‘‘human-interest’’ or ‘‘news you can use.’’
\end{quote}
Dabei lässt sich Soft News als Obergruppe von Clickbait verstehen, da es die Merkmale Sensationsmacherei, Menschliches Interesse und News die man gebrauchen kann teilt, jedoch spezifischer ist in der Hinsicht das Clickbait sich primär durch den Überschriftenstil auszeichnet.
Soft News zeigt sich in zu zunehmender Sensationsmacherei, Aufmerksamkeit für  \enquote{Human Interest}, Kriminalität und Katastrophen widerspiegelt. Auch zeigen sie einen Anstieg von Selbstreferenz in News, ein für Clickbait typisches Mittel. 

Dor beschreibt den unterschied zwischen traditionellen Überschriften und Boulevardzeitungsüberschriften. Erstere sind demnach üblicherweise kurze, telegrammartige Zusammenfassungen, oder Zitate. Boulevardzeitungsüberschriften machen dies nicht, sie sind üblicherweise nicht informativ, sondern haben die Aufgabe Weltanschaungs- und Glaubenssysteme im Leser anzusteuern. In der Arbeit werden folgende Regeln aufgestellt, wie eine Überschrift gestaltet sein sollte:
\begin{itemize}
\item Headlines should be as short as possible
\item Headlines should be clear, easy to understand, and unambiguous
\item Headlines should be interesting
\item Headlines should contain new information
\item Headlines should not presuppose information unknown to the readers
\item Headlines should include names and concepts with high 'news value' for the readers
\item Headlines should not contain names and concepts with low 'news value'
\item Headlines should connect
\item Headlines should 'connect the story' to prior expectations and assumptions
\item Headlines should 'the story in a appropriate fashion
\end{itemize}
Diese Regeln decken sich mit dem früher eingeführten Konzept der Hard News, und stellen in dieser Arbeit den Gegenentwurf zu Clickbait dar. 
Es werden typische Attribute von Boulevardzeitungsüberschriften formuliert:
\begin{itemize}
\item Cognitive effort AND new information minimized
\item Maximize context of interpretation
\item Low information -> many questions
\item Extremely rich context of interpretation 
\item Cliches
\item Prejudices
\item Fear
\item Passion + hatred
\end{itemize}
Diese haben Ähnlichkeiten mit Clickbait, entsprechen jedoch nicht genau der Definition, insbesondere aufgrund der fehlenden Betonung der clickbaittypischen Vorwärtsreferenzierung. Jedoch lassen sich die Punkte 1,3,4,7,8 auch oft in Clickbait finden.

% subsection relevance_theory (end)
% section soft_und_hard_news (end)

\cite{discours-8600-10-understanding-how-headings-influence-text-processing}

\section{Überschriften} % (fold)
\label{sec:_berschriften}
\subsection{Relevance Theory} % (fold)
\label{sub:relevance_theory}
Relevance Theory.
Relevance Theory \cite{wilson2002relevance} beschreibt Relevanz für Menschen als eine Funktion aus Verarbeitungsaufwand und positivem kognitivem Effekt. Sie behauptet das der menschliche kognitive Prozess darauf ausgelegt ist den größtmöglichen positiven kognitiven Effekt für den kleinstmöglichen Verarbeitungsaufwand zu erreichen.
Darauf basierend folgert Dor in \enquote{On newspaper headlines as relevance optimizers} \cite{dor2003newspaper} das Zeitungsüberschriften Relevanzoptimierer sind, also das sie gestaltet sind um für den Leser möglichst relevant zu erscheinen. 

% section _berschriften (end)

\section{Definition Clickbait} 
\label{sec:definition_clickbait}

Merriam Webster \cite{merriam-webster:clickbait}
\quote{
  :  something (such as a headline) designed to make readers want to click on a hyperlink especially when the link leads to content of dubious value or interest <It is difficult to remember a time when you could scroll through the social media outlet of your choice and not be bombarded with: You'll never believe what happened when … This is the cutest thing ever … This is the biggest mistake you can make … Take this quiz to see which character you are on … They are all classic clickbait models. And they are irritating as hell. There's no singular way to craft clickbait, but the essence is clear: Lure—no trick—readers to your site. — Emily Shire, Daily Beast, 14 July 2014> < … “clickbait,” those seductive Huffington Post-esque headlines that suck up your attention but don't deliver what they promise? — Oliver Burkeman, The Guardian (London), 10 Aug. 2013> < … there's an incentive to combine clickbait, to get people in, with strong content to keep them on the site. — Steve Hind, interviewed on National Public Radio, 10 Nov. 2013>
}
Wikipedia \cite{wikipedia:clickbait}

% section definition_clickbait (end)

\section{Motivation} 
\label{sec:motivation}




% section motivation (end)

\section{Problematik} 
\label{sec:problematik}

% section problematik (end)

\section{Meinungen} 
\label{sec:meinungen}

% section meinungen (end)

% section relat (end)
% chapter clickbait (end)
